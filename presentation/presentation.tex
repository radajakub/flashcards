\documentclass[11pt, xcolor=table]{beamer}
\usepackage[utf8]{inputenc}
\usepackage[czech]{babel}
\usepackage{graphicx}
\usepackage{pifont}
\usepackage{listings}
\usepackage{xcolor}
\usepackage{hyperref}

\beamertemplatenavigationsymbolsempty

\graphicspath{ {resources/} }

\usetheme{Madrid}
\usecolortheme{seagull}

\title{Flashcards}
\subtitle{Maturitní práce}
\author{Jakub Rada O8.B}
\institute{Gymnázium Nad Alejí}
\date{16. května 2019}

\begin{document}
%title page
    \frame{\titlepage}
%about project
    \begin{frame}
        \frametitle{O aplikaci}
        Flashcards je aplikace na učení a testování slovíček či jiné látky
        \begin{itemize}
            \item Vytváření kartiček, které se dají zařadit do okruhů ke zkoušení
            \item Tři typy testů z okruhů
            \begin{itemize}
                \item Procházení a otáčení kartiček
                \item Vybírání správné možnosti
                \item Psaní správné odpovědi
            \end{itemize}
            \item Rychlejší vytváření kartiček a sdílení mezi zařízeními
            \begin{itemize}
                \item Import do .csv souboru
                \item Export z .csv souboru do aplikace
            \end{itemize}
        \end{itemize}
    \end{frame}
%motivation frame
    \begin{frame}
        \frametitle{Motivace}
        \begin{itemize}
            \item Několik podobných aplikací existuje (např. Quizlet) \ding{219} všechny online
            \item Tato vytvořená aplikace nevyžaduje připojení k internetu
            \item Sdílení dat lze uskutečnit skrze .csv soubory
        \end{itemize}
        \vfill
        \begin{center}
            \url{https://quizlet.com/}
        \end{center}
    \end{frame}
%show slide
    \begin{frame}
        \frametitle{Ukázka}
    \end{frame}
%levenshtein distance
    \begin{frame}
        \frametitle{Jak najít chyby v psané odpovědi?}
        Levenshteinova vzdálenost
        \begin{itemize}
            \item porovnáváme 2 textové řetězce
            \item tři možné operace - přidání znaku, vynechání znaku, záměna znaku
        \end{itemize}
        \begin{table}
            \centering
            \label{tab:1}
            \begin{tabular}{ | c | c | c | c | c | c | c | }
                \hline
                &   & j & e & d & n & a \\
                \hline
                & \cellcolor{lightgray}0 & 1 & 2 & 3 & 4 & 5 \\
                \hline
                j & 1 & \cellcolor{lightgray}0 & 1 & 2 & 3 & 4 \\
                \hline
                e & 2 & 1 & \cellcolor{lightgray}0 & 1 & 2 & 3 \\
                \hline
                m & 3 & 2 & 1 & \cellcolor{lightgray}1 & 2 & 3 \\
                \hline
                n & 4 & 3 & 2 & 2 & \cellcolor{lightgray}1 & 2 \\
                \hline
                a & 5 & 4 & 3 & 3 & 2 & \cellcolor{lightgray}1 \\
                \hline
            \end{tabular}
        \end{table}
    \end{frame}
%card collision frame
    \begin{frame}
        \frametitle{Kartičky, které mají stejnou jednu stranu}
        Kartičky se stejnou přední stranou
        \begin{itemize}
            \item Kartičky se po dobu testu spojí do jedné
            \begin{itemize}
                \item Ve vybírání odpovědí vznikne jedna možnost
                \item V psaní odpovědi stačí odpovědět jakoukoli částí
            \end{itemize}
            \bigskip
            $$\frac{orange}{oranzova}\, a\,\frac{orange}{pomeranc}\,\rightarrow\,\frac{orange}{oranzova, pomeranc}$$
        \end{itemize}
    \end{frame}
%import slide
    \begin{frame}[fragile]
        \frametitle{Import a export}
        \begin{itemize}
            \item Formát zápisu dat do .csv souboru
            \begin{lstlisting}
card, lic, rub, cislo_okruhu|cislo_okruhu
tag, cislo_okruhu, nazev_okruhu
            \end{lstlisting}
            \vfill
            \item Ukázkový soubor
            \begin{lstlisting}
tag, 1, cisla
tag, 2, anglictina
card, jedna, one, 1|2
card, dva, two, 1|2
card, tri, tres, 1
            \end{lstlisting}
        \end{itemize}
    \end{frame}
%sources
    \begin{frame}
        \frametitle{Použité knihovny}
        Knihovny
        \begin{itemize}
            \item server - Python 3, Django
            \item klient - Electron, Bootstrap, JQuery
            \item NPM
        \end{itemize}
        Dokumentace
        \begin{itemize}
            \item Electron docs
            \item Django docs
            \item Bootstrap docs
        \end{itemize}
    \end{frame}
%end page
    \begin{frame}
        \frametitle{Konec}
        \begin{block}{}
            \centering
            {\Huge Děkuji za pozornost}
        \end{block}
    \end{frame}
\end{document}